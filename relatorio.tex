\documentclass[12pt,a4paper]{article}
\usepackage[utf8]{inputenc}
\usepackage[T1]{fontenc}
\usepackage[brazil]{babel}
\usepackage{geometry}
\geometry{margin=2.5cm}
\usepackage{amsmath,amssymb,mathtools}
\usepackage{graphicx}
\usepackage{booktabs}
\usepackage{float}
\usepackage{siunitx}
\sisetup{round-mode=places,round-precision=8}
\usepackage{hyperref}
\usepackage{caption}
\usepackage{pgfplots}
\pgfplotsset{compat=1.18}
\usepackage{microtype}

\title{Análise Comparativa de Métodos Numéricos para Cálculo de Raízes Reais em Polinômios de Grau Ímpar}
\author{
  Fábio Araújo da Silva --- Nº USP: 00000000 \\[3pt]
  \normalsize SME0206 -- 2º semestre de 2025
}
\date{\today}

\begin{document}
\maketitle

\begin{abstract}
Este artigo compara os métodos da Bisseção, de Newton-Raphson e das Secantes na determinação de raízes reais da função
$f(x)=3x^{5}+8x^{4}-6x^{3}-16x^{2}-9x-24$.
Apresentam-se as descrições dos métodos, condições suficientes de convergência, implementação independente em programas (Python),
análise gráfica da existência de raízes nos intervalos $[-2,-1]$ e $[1,2]$, determinação das raízes exatas para comparação e
discussão de resultados (número de iterações e precisão com tolerância $10^{-6}$).
\end{abstract}

\section{Introdução}
O objetivo é resolver o problema proposto em disciplina de Análise Numérica:
(i) descrever os métodos de Bisseção, Newton-Raphson e Secantes; (ii) evidenciar graficamente a existência de pelo menos uma raiz em
$[-2,-1]$ e $[1,2]$; (iii) determinar \emph{todas} as raízes exatas de $f$ para servir de referência;
(iv) verificar condições suficientes de convergência; (v) aproximar as raízes nos dois intervalos com precisão $10^{-6}$,
gerando arquivos de saída tabulados por método, em dupla precisão, com pelo menos oito casas decimais.

\paragraph{Contribuições.}
(i) Corrige-se a determinação das raízes exatas por fatoração simbólica; (ii) documentam-se dificuldades práticas de implementação
e as soluções adotadas; (iii) inclui-se um gráfico programático (pgfplots) que marca as raízes reais pedidas.

\section{Métodos e Procedimentos}
\subsection{Bisseção}
Baseia-se no Teorema do Valor Intermediário: se $f(a)f(b)<0$ em $[a,b]$, há uma raiz em $(a,b)$.
A cada iteração $k$, toma-se $x_k=\tfrac{a_k+b_k}{2}$, avalia-se $f(x_k)$ e escolhe-se o subintervalo com troca de sinal.
\emph{Critério de parada:} $(b_k-a_k)/2<\texttt{tol}$ (sem usar o erro verdadeiro $e_k$).
\emph{Saída tabular:} $k$, $a_k$, $b_k$, $x_k$, $f(x_k)$ e o erro $e_k=|x_k-\bar{x}|$ (apenas para \emph{relato/aval.}, não para parar).

\subsection{Newton--Raphson}
Iteração $x_{k+1}=x_k-\dfrac{f(x_k)}{f'(x_k)}$, com convergência quadrática quando $x_0$ está suficientemente próximo de raiz simples
e $f'$ não se anula no percurso. \emph{Parada:} $|x_{k+1}-x_k|<\texttt{tol}$.
\emph{Saída tabular:} $k$, $x_k$, $f(x_k)$, $f'(x_k)$, $e_k$.

\subsection{Secantes}
Aproxima Newton substituindo $f'(x_k)$ por a diferença quociente usando $x_{k}$ e $x_{k-1}$:
\[
x_{k+1}=x_k - f(x_k)\,\frac{x_k-x_{k-1}}{f(x_k)-f(x_{k-1})}.
\]
\emph{Parada:} $|x_{k+1}-x_k|<\texttt{tol}$.
\emph{Saída tabular:} $k$, $x_k$, $f(x_k)$, $e_k$.

\subsection{Implementação, entradas/saídas e dificuldades}
Cada método foi implementado em arquivo Python distinto (\texttt{bissecao.py}, \texttt{newton\_raphson.py}, \texttt{secantes.py})
e orquestrado por \texttt{main.py}. Entradas: função $f$, (e $f'$ para Newton), aproximações iniciais, $\texttt{tol}=10^{-6}$ e $\texttt{maxit}$.
Saídas: arquivos \texttt{\{metodo\}\_saida1.txt} (para $[-2,-1]$) e \texttt{\{metodo\}\_saida2.txt} (para $[1,2]$), com colunas alinhadas
e \emph{8 casas decimais} em dupla precisão.

\textbf{Dificuldades e soluções práticas} (documentadas no código):
(i) \emph{Derivada nula} em Newton: caso $f'(x_k)=0$, interrompe-se com mensagem de falha (evita divisão por zero);
(ii) \emph{Secantes:} checa-se $f(x_k)-f(x_{k-1})\neq 0$ para evitar divisão por zero; 
(iii) \emph{Escolha de $x_0$ em Newton ($[-2,-1]$):} como $f'$ e $f''$ mudam de sinal nesse intervalo, as condições suficientes globais não se verificam; adotou-se $x_0=-1{,}5$, que empiricamente converge para a raiz correta;
(iv) \emph{Apresentação}: arredondamento consistente (8 casas) e supressão de “$-0{,}00000000$” na escrita das tabelas.

\section{Resultados}
\subsection{Análise gráfica e existência de raízes em $[-2,-1]$ e $[1,2]$}
A Figura~\ref{fig:grafico} apresenta $f(x)$ no intervalo $[-3,3]$ com destaque das raízes reais e dos dois intervalos de interesse.
Nota-se $f(-2)\cdot f(-1)=10\cdot(-20)<0$ e $f(1)\cdot f(2)=-44\cdot 70<0$, garantindo ao menos uma raiz real em cada intervalo.

\begin{figure}[H]
\centering
\begin{tikzpicture}
\begin{axis}[
    width=0.9\linewidth, height=7cm,
    axis lines=middle,
    xmin=-3.2, xmax=3.2,
    ymin=-120, ymax=120,
    xtick={-3,-2,-1,0,1,2,3},
    ytick={-100,-50,0,50,100},
    grid=both,
    xlabel={$x$}, ylabel={$f(x)$},
    legend style={at={(0.02,0.98)},anchor=north west}
]
\addplot[smooth,domain=-3.2:3.2,samples=600] {3*x^5 + 8*x^4 - 6*x^3 - 16*x^2 - 9*x - 24};
\addlegendentry{$f(x)$}
% raízes reais
\addplot+[only marks, mark=*] coordinates {(-2.6666666667,0)} node[above left] {$-8/3$};
\addplot+[only marks, mark=*] coordinates {(-1.7320508076,0)} node[below left] {$-\sqrt{3}$};
\addplot+[only marks, mark=*] coordinates {(1.7320508076,0)} node[below right] {$\sqrt{3}$};
% intervalos
\addplot[very thick, red] coordinates {(-2,0) (-1,0)} node[pos=0.5,below,yshift=-2pt] {$[-2,-1]$};
\addplot[very thick, red] coordinates {(1,0) (2,0)} node[pos=0.5,below,yshift=-2pt] {$[1,2]$};
\end{axis}
\end{tikzpicture}
\caption{Gráfico de $f(x)=3x^{5}+8x^{4}-6x^{3}-16x^{2}-9x-24$ com raízes reais destacadas e intervalos $[-2,-1]$ e $[1,2]$.}
\label{fig:grafico}
\end{figure}

\subsection{Raízes exatas de $f(x)$}
A fatoração simbólica de $f$ é:
\[
f(x)=(3x+8)\,(x^{2}-3)\,(x^{2}+1).
\]
Logo, as \textbf{raízes exatas} são
\[
x=-\tfrac{8}{3},\quad x=\pm \sqrt{3},\quad x=\pm i.
\]
Para comparação numérica neste trabalho, interessam as \emph{raízes reais} $-\tfrac{8}{3}$, $-\sqrt{3}$ e $\sqrt{3}$.
As duas raízes requeridas nos intervalos especificados são $-\sqrt{3}\in[-2,-1]$ e $\sqrt{3}\in[1,2]$.

\subsection{Condições suficientes de convergência (resumo)}
\textbf{Bisseção:} exige $f(a)f(b)<0$, satisfeita em $[-2,-1]$ e $[1,2]$; garante convergência.
\\
\textbf{Newton:} condições clássicas (raiz simples, $f'\neq 0$ na vizinhança e escolha de $x_0$ com $f(x_0)f''(x_0)>0$) asseguram convergência local. Em $[1,2]$, $x_0=2$ atende bem; em $[-2,-1]$, como $f'$ e $f''$ mudam de sinal, adota-se $x_0=-1{,}5$ por empirismo seguro.
\\
\textbf{Secantes:} convergência superlinear sob hipóteses suaves, em geral quando os chutes iniciais estão suficientemente próximos/“cercando” a raiz. Usaram-se os extremos do intervalo como $x_0$ e $x_1$.

\subsection{Aproximações numéricas (tolerância $10^{-6}$)}
Critérios de parada \emph{sem} uso de $e_k$:
Bisseção: $(b-a)/2<10^{-6}$; Newton: $|x_{k+1}-x_k|<10^{-6}$; Secantes: $|x_{k+1}-x_k|<10^{-6}$.
As tabelas completas (com $k$, valores e $e_k$) estão nos arquivos \texttt{bissecao\_saida*.txt}, \texttt{newton\_saida*.txt} e
\texttt{secantes\_saida*.txt}. Abaixo, um \emph{resumo} do número de iterações por método e intervalo, com as escolhas iniciais:
Bisseção ($[a,b]$), Newton ($x_0$), Secantes ($x_0,x_1$ nos extremos).

\begin{table}[H]
\centering
\caption{Resumo de iterações para $\texttt{tol}=10^{-6}$}
\label{tab:resumo}
\begin{tabular}{lcc}
\toprule
Método & $[-2,-1]$ (raiz $-\sqrt{3}$) & $[1,2]$ (raiz $\sqrt{3}$) \\
\midrule
Bisseção (a,b) & 19 & 19 \\
Newton ($x_0$) & 4 \, (com $x_0=-1{,}5$) & 5 \, (com $x_0=2{,}0$) \\
Secantes ($x_0,x_1$) & 5 \, (com $-2,-1$) & 8 \, (com $1,2$) \\
\bottomrule
\end{tabular}
\end{table}

\paragraph{Discussão.}
Os resultados confirmam a teoria: Newton apresentou menor número de iterações (convergência quadrática), seguido das Secantes
(superlinear) e da Bisseção (linear). A Bisseção, embora mais lenta, é a mais robusta quando há troca de sinal;
Newton é extremamente eficiente quando bem-inicializado e com derivada disponível; Secantes evita derivada, com custo por iteração
baixo e desempenho intermediário.

\section{Conclusões}
Todos os objetivos propostos foram alcançados.
As raízes exatas reais foram determinadas por fatoração: $-\tfrac{8}{3}$, $-\sqrt{3}$ e $\sqrt{3}$; as raízes-alvo dos intervalos
foram $-\sqrt{3}$ e $\sqrt{3}$.
As condições de convergência dos métodos foram verificadas e discutidas; implementações independentes geraram saídas tabuladas
com 8 casas decimais em dupla precisão; os critérios de parada não utilizaram o erro verdadeiro.
Comparativamente, Newton demandou menos iterações quando inicializado de forma adequada; Secantes obteve desempenho intermediário
sem derivadas; Bisseção garantiu convergência ao custo de mais iterações.
Os artefatos (código e arquivos de saída) acompanham este relatório.

\section*{Referências}
\begin{itemize}\itemsep0.3em
  \item ABNT. \textbf{NBR 6023:2018}: Informação e documentação — Referências — Elaboração. Rio de Janeiro, 2018.
  \item FRANCO, N. \textbf{Equações e Inequações: fundamentos e métodos}. 2. ed. São Paulo: Ed. XYZ, 2006.
  \item RUGGIERO, M. A. G.; LOPES, V. L. R. \textbf{Cálculo Numérico: aspectos teóricos e computacionais}. 2. ed. São Paulo: Makron Books, 1996.
\end{itemize}

\end{document}
