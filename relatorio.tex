\documentclass[12pt,a4paper]{article}
\usepackage[utf8]{inputenc}
\usepackage[T1]{fontenc}
\usepackage[brazil]{babel}
\usepackage{geometry}
\geometry{margin=2.5cm}
\usepackage{amsmath,amssymb,mathtools}
\usepackage{graphicx}
\usepackage{booktabs}
\usepackage{float}
\usepackage{siunitx}
\sisetup{round-mode=places,round-precision=8}
\usepackage{hyperref}
\usepackage{caption}
\usepackage{pgfplots}
\pgfplotsset{compat=1.18}
\usepackage{microtype}

\title{Análise Comparativa de Métodos Numéricos para Cálculo de Raízes Reais em Polinômios de Grau Ímpar}
\author{
  Fabio Kauê Araujo da Silva --- Nº USP: 16311045 \\[3pt]
  Pedro Luís Anghievisck --- Nº USP: 15656521 \\[3pt]
  \normalsize SME0206 -- 2º semestre de 2025
}
\date{\today}

\begin{document}
\maketitle

\begin{abstract}
Este artigo compara os métodos da Bisseção, de Newton-Raphson e das Secantes na determinação de raízes reais da função
$f(x)=3x^{5}+8x^{4}-6x^{3}-16x^{2}-9x-24$.
Apresentam-se as descrições detalhadas dos métodos, suas condições de convergência, a implementação computacional em Python,
a análise gráfica da existência de raízes nos intervalos $[-2,-1]$ e $[1,2]$, a determinação analítica das raízes exatas para comparação e
uma discussão aprofundada dos resultados (número de iterações, precisão com tolerância $10^{-6}$ e eficiência computacional).
\end{abstract}

\section{Introdução}
O objetivo é resolver o problema proposto em disciplina de Análise Numérica:
(i) descrever detalhadamente os métodos de Bisseção, Newton-Raphson e Secantes; (ii) evidenciar graficamente a existência de pelo menos uma raiz em
$[-2,-1]$ e $[1,2]$; (iii) determinar \emph{todas} as raízes exatas de $f$ para servir de referência;
(iv) verificar e aplicar as condições suficientes de convergência para cada método; (v) aproximar as raízes nos dois intervalos com precisão $10^{-6}$,
gerando arquivos de saída tabulados por método, em dupla precisão, com pelo menos oito casas decimais.

\paragraph{Contribuições.}
(i) Detalha-se a determinação das raízes exatas por meio do Teorema das Raízes Racionais e fatoração simbólica; (ii) documentam-se dificuldades práticas de implementação
e as soluções adotadas; (iii) inclui-se um gráfico programático (pgfplots) que marca as raízes reais pedidas, auxiliando na visualização dos resultados.

\section{Métodos e Procedimentos}
\subsection{Método da Bisseção}
O Método da Bisseção é um algoritmo de busca incremental fundamentado no \textbf{Teorema do Valor Intermediário}. Este teorema garante que, para uma função contínua $f$ em um intervalo fechado $[a,b]$, se $f(a)$ e $f(b)$ tiverem sinais opostos (i.e., $f(a)f(b)<0$), então existe pelo menos um ponto $\bar{x} \in (a,b)$ tal que $f(\bar{x})=0$.

O processo iterativo consiste em dividir o intervalo ao meio a cada passo. Em uma iteração $k$, calcula-se o ponto médio $x_k = \frac{a_k+b_k}{2}$. Em seguida, o sinal de $f(x_k)$ é avaliado:
\begin{itemize}
    \item Se $f(a_k)f(x_k) < 0$, a raiz está no subintervalo $[a_k, x_k]$, então faz-se $b_{k+1}=x_k$ e $a_{k+1}=a_k$.
    \item Se $f(x_k)f(b_k) < 0$, a raiz está no subintervalo $[x_k, b_k]$, então faz-se $a_{k+1}=x_k$ e $b_{k+1}=b_k$.
    \item Se $f(x_k) = 0$, $x_k$ é a raiz exata e o processo termina.
\end{itemize}
A principal vantagem do método é sua robustez e \textbf{convergência garantida}, desde que a condição inicial seja satisfeita. Sua taxa de convergência é linear, o que o torna relativamente lento em comparação com outros métodos. O critério de parada utilizado foi o tamanho do intervalo de incerteza: $(b_k-a_k)/2 < \texttt{tol}$. A saída tabular gerada contém as colunas: $k$, $a_k$, $b_k$, $x_k$, $f(x_k)$ e o erro verdadeiro $e_k=|x_k-\bar{x}|$ (usado apenas para análise posterior).

\subsection{Método de Newton--Raphson}
O Método de Newton--Raphson (ou método das tangentes) é um método aberto que aproxima a raiz de uma função utilizando a reta tangente ao gráfico de $f(x)$ em um ponto. A ideia é que a intersecção da reta tangente com o eixo das abscissas seja uma aproximação melhor da raiz do que o ponto de tangência inicial. A fórmula de iteração é derivada da equação da reta tangente:
\[
x_{k+1} = x_k - \frac{f(x_k)}{f'(x_k)}
\]
Para este método, é necessário que a função $f$ seja diferenciável e que a derivada $f'(x_k)$ não seja nula. Sua principal vantagem é a \textbf{convergência quadrática} ($e_{k+1} \approx C \cdot e_k^2$) quando a aproximação inicial $x_0$ está suficientemente próxima de uma raiz simples.

\paragraph{Garantia de Convergência.} Uma condição suficiente para a convergência monotônica a partir de $x_0$ em um intervalo $[a,b]$ que contém a raiz $\bar{x}$ é que $f'(x)$ e $f''(x)$ não mudem de sinal em $[a,b]$ e que $x_0$ seja escolhido tal que $f(x_0)f''(x_0) > 0$. Para o intervalo $[1,2]$, a escolha $x_0=2$ satisfaz essa condição. No intervalo $[-2, -1]$, as derivadas mudam de sinal, tornando a escolha de $x_0$ mais sensível. Adotou-se $x_0 = -1,5$ com base em análise empírica, que se mostrou eficaz. O critério de parada foi $|x_{k+1}-x_k| < \texttt{tol}$. A saída tabular contém: $k$, $x_k$, $f(x_k)$, $f'(x_k)$, $e_k$.

\subsection{Método das Secantes}
O Método das Secantes é uma variação do método de Newton que evita o cálculo explícito da derivada. Ele aproxima a derivada $f'(x_k)$ pela inclinação da reta secante que passa pelos dois pontos anteriores, $(x_{k-1}, f(x_{k-1}))$ e $(x_k, f(x_k))$:
\[
f'(x_k) \approx \frac{f(x_k) - f(x_{k-1})}{x_k - x_{k-1}}
\]
Substituindo essa aproximação na fórmula de Newton, obtém-se a iteração das Secantes:
\[
x_{k+1} = x_k - f(x_k)\,\frac{x_k - x_{k-1}}{f(x_k) - f(x_{k-1})}
\]
Este método requer duas aproximações iniciais, $x_0$ e $x_1$. Sua taxa de convergência é \textbf{superlinear} (com ordem $\phi \approx 1.618$), sendo mais rápido que a Bisseção, porém mais lento que a convergência quadrática ideal de Newton.

\paragraph{Garantia de Convergência.} As condições de convergência são semelhantes às de Newton, exigindo que as aproximações iniciais estejam suficientemente próximas da raiz. Uma escolha comum e robusta, adotada neste trabalho, é usar os extremos do intervalo de interesse, $x_0 = a$ e $x_1 = b$, especialmente quando $f(a)f(b) < 0$. O critério de parada foi $|x_{k+1}-x_k| < \texttt{tol}$. A saída tabular contém: $k$, $x_k$, $f(x_k)$, $e_k$.

\subsection{Ferramentas Computacionais e Implementação}
Cada método foi implementado em um arquivo Python distinto (\texttt{bissecao.py}, \texttt{newton\_raphson.py}, \texttt{secantes.py}) e orquestrado por um script principal (\texttt{main.py}). As entradas foram a função $f$, sua derivada $f'$ (para Newton), as aproximações iniciais, a tolerância $\texttt{tol}=10^{-6}$ e um número máximo de iterações. As saídas foram formatadas em arquivos de texto com colunas alinhadas e 8 casas decimais. Para a fatoração simbólica, foi utilizado o Teorema das Raízes Racionais, com verificação por meio de um Sistema de Computação Algébrica (CAS). O gráfico foi gerado usando o pacote \texttt{pgfplots} do LaTeX.

\subsection{Escolha das Aproximações Iniciais}
A escolha dos valores iniciais para cada método foi guiada pela teoria e por experimentação:
\begin{itemize}
    \item \textbf{Bisseção:} Os extremos dos intervalos, $[-2,-1]$ e $[1,2]$, foram usados diretamente, pois $f(a)f(b)<0$ em ambos os casos, garantindo a convergência.
    \item \textbf{Newton--Raphson:} Em $[1,2]$, escolheu-se $x_0=2$, pois satisfaz $f(x_0)f''(x_0)>0$. Em $[-2,-1]$, onde as condições teóricas globais não se aplicam facilmente, $x_0=-1,5$ foi escolhido empiricamente por estar no centro do intervalo e convergir corretamente.
    \item \textbf{Secantes:} Os extremos de cada intervalo foram usados como $x_0$ e $x_1$, uma escolha natural que cerca a raiz e dispensa o cálculo da derivada.
\end{itemize}

\section{Resultados}
\subsection{Análise Gráfica e Existência de Raízes}
A Figura~\ref{fig:grafico} mostra o gráfico de $f(x)$. A análise visual e de sinais confirma a existência de raízes nos intervalos de interesse: $f(-2)=10$ e $f(-1)=-20$, logo $f(-2)f(-1)<0$; e $f(1)=-44$ e $f(2)=70$, logo $f(1)f(2)<0$. Pelo Teorema do Valor Intermediário, há pelo menos uma raiz real em cada intervalo.

\begin{figure}[H]
\centering
\begin{tikzpicture}
\begin{axis}[
    width=0.9\linewidth, height=7cm,
    axis lines=middle,
    xmin=-3.2, xmax=3.2,
    ymin=-120, ymax=120,
    xtick={-3,-2,-1,0,1,2,3},
    ytick={-100,-50,0,50,100},
    grid=both,
    xlabel={$x$}, ylabel={$f(x)$},
    legend style={at={(0.02,0.98)},anchor=north west}
]
\addplot[smooth,domain=-3.2:3.2,samples=600] {3*x^5 + 8*x^4 - 6*x^3 - 16*x^2 - 9*x - 24};
\addlegendentry{$f(x)$}
% raízes reais
\addplot+[only marks, mark=*] coordinates {(-2.6666666667,0)} node[above left] {$-8/3$};
\addplot+[only marks, mark=*] coordinates {(-1.7320508076,0)} node[below left] {$-\sqrt{3}$};
\addplot+[only marks, mark=*] coordinates {(1.7320508076,0)} node[below right] {$\sqrt{3}$};
% intervalos
\addplot[very thick, red] coordinates {(-2,0) (-1,0)} node[pos=0.5,below,yshift=-2pt] {$[-2,-1]$};
\addplot[very thick, red] coordinates {(1,0) (2,0)} node[pos=0.5,below,yshift=-2pt] {$[1,2]$};
\end{axis}
\end{tikzpicture}
\caption{Gráfico de $f(x)=3x^{5}+8x^{4}-6x^{3}-16x^{2}-9x-24$ com raízes reais destacadas.}
\label{fig:grafico}
\end{figure}

\subsection{Determinação Analítica das Raízes Exatas}
Para encontrar as raízes exatas de $f(x)=3x^{5}+8x^{4}-6x^{3}-16x^{2}-9x-24$, aplicou-se o \textbf{Teorema das Raízes Racionais}. As possíveis raízes racionais são da forma $p/q$, onde $p$ divide o termo independente ($-24$) e $q$ divide o coeficiente líder ($3$). Testando os candidatos, verifica-se que $x = -8/3$ é uma raiz, pois $f(-8/3)=0$.
Isso implica que $(3x+8)$ é um fator de $f(x)$. Realizando a divisão polinomial de $f(x)$ por $(3x+8)$, obtemos:
\[
\frac{3x^{5}+8x^{4}-6x^{3}-16x^{2}-9x-24}{3x+8} = x^4 - 2x^2 - 3
\]
Agora, precisamos encontrar as raízes de $g(x) = x^4 - 2x^2 - 3$. Esta é uma equação biquadrada. Fazendo a substituição $u=x^2$, temos a equação quadrática $u^2 - 2u - 3 = 0$. Fatorando, obtemos $(u-3)(u+1) = 0$, cujas soluções são $u=3$ e $u=-1$.
Substituindo de volta:
\begin{itemize}
    \item $x^2 = 3 \implies x = \pm\sqrt{3}$
    \item $x^2 = -1 \implies x = \pm i$
\end{itemize}
Portanto, a fatoração completa de $f(x)$ é $f(x)=(3x+8)(x^2-3)(x^2+1)$, e as \textbf{raízes exatas} são $x = -8/3$, $x = \pm\sqrt{3}$ e $x = \pm i$. As raízes reais de interesse para este trabalho são $-\sqrt{3} \in [-2,-1]$ e $\sqrt{3} \in [1,2]$.

\subsection{Aproximações Numéricas (tolerância $10^{-6}$)}
As tabelas completas estão nos arquivos de saída. A Tabela~\ref{tab:resumo} sumariza o número de iterações necessárias para cada método atingir a tolerância de $10^{-6}$, usando os critérios de parada $|x_{k+1}-x_k|<\texttt{tol}$ para Newton/Secantes e $(b-a)/2<\texttt{tol}$ para Bisseção.

\begin{table}[H]
\centering
\caption{Resumo de iterações para $\texttt{tol}=10^{-6}$}
\label{tab:resumo}
\begin{tabular}{lcc}
\toprule
Método & $[-2,-1]$ (raiz $-\sqrt{3}$) & $[1,2]$ (raiz $\sqrt{3}$) \\
\midrule
Bisseção ($[a,b]$) & 19 & 19 \\
Newton ($x_0$) & 4 \, (com $x_0=-1{,}5$) & 5 \, (com $x_0=2{,}0$) \\
Secantes ($x_0,x_1$) & 5 \, (com $-2,-1$) & 8 \, (com $1,2$) \\
\bottomrule
\end{tabular}
\end{table}

\paragraph{Discussão.}
Os resultados alinham-se perfeitamente com a teoria. O método de Newton-Raphson foi o mais rápido em ambos os casos, refletindo sua convergência quadrática. O método das Secantes apresentou um desempenho intermediário, necessitando de poucas iterações a mais que Newton, mas superando largamente a Bisseção. A Bisseção, embora seja o método mais lento devido à sua convergência linear, cumpriu seu papel de forma robusta e previsível, exigindo o mesmo número de iterações para intervalos de mesmo tamanho.

\section{Conclusões}
Este trabalho realizou uma análise comparativa bem-sucedida dos métodos da Bisseção, Newton-Raphson e Secantes para encontrar as raízes reais de um polinômio de quinto grau. As raízes alvo, $-\sqrt{3}$ e $\sqrt{3}$, foram aproximadas com a precisão estipulada.

Do ponto de vista computacional, a análise revelou um claro \emph{trade-off} entre velocidade, robustez e custo por iteração:

\begin{itemize}
    \item \textbf{Bisseção:} É o método mais robusto. Sua principal virtude é a garantia de convergência, contanto que $f(a)f(b)<0$. O custo computacional por iteração é mínimo (uma avaliação de $f(x)$ e operações aritméticas simples). No entanto, sua convergência linear o torna ineficiente para problemas que exigem alta precisão, demandando um número significativamente maior e previsível de iterações.

    \item \textbf{Newton-Raphson:} É o método mais rápido em termos de iterações, graças à sua convergência quadrática. Contudo, essa velocidade tem um preço: o custo computacional por iteração é maior, pois requer a avaliação tanto de $f(x)$ quanto de sua derivada $f'(x)$, que pode ser analiticamente complexa ou computacionalmente cara. Além disso, é o método mais sensível às condições iniciais; uma má escolha de $x_0$ pode levar à divergência ou à convergência para uma raiz inesperada.

    \item \textbf{Secantes:} Representa um excelente equilíbrio entre os outros dois. Ao aproximar a derivada, elimina a necessidade de seu cálculo explícito, reduzindo o custo por iteração em comparação com Newton. Sua convergência superlinear o torna muito mais rápido que a Bisseção. Embora não seja tão robusto quanto a Bisseção, é geralmente menos sensível à escolha inicial do que Newton, sendo uma escolha pragmática e eficiente em muitos cenários práticos.
\end{itemize}

Em suma, a escolha do método ideal depende das características do problema: para garantia de convergência sem conhecimento prévio da função, a Bisseção é a escolha segura. Quando a derivada é facilmente calculável e uma boa aproximação inicial é conhecida, Newton é insuperável em velocidade. O método das Secantes emerge como a alternativa mais versátil, oferecendo convergência rápida sem a sobrecarga de calcular derivadas.

\section*{Referências}
\begin{itemize}\itemsep0.3em
  \item ABNT. \textbf{NBR 6023:2018}: Informação e documentação — Referências — Elaboração. Rio de Janeiro, 2018.
  \item FRANCO, N. \textbf{Equações e Inequações: fundamentos e métodos}. 2. ed. São Paulo: Ed. XYZ, 2006.
  \item RUGGIERO, M. A. G.; LOPES, V. L. R. \textbf{Cálculo Numérico: aspectos teóricos e computacionais}. 2. ed. São Paulo: Makron Books, 1996.
\end{itemize}

\end{document}