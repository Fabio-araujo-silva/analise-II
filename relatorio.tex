\documentclass[12pt, a4paper]{article}

% PACOTES BÁSICOS
\usepackage[utf8]{inputenc}
\usepackage[T1]{fontenc}
\usepackage[brazil]{babel}
\usepackage{geometry}
\usepackage{amsmath}
\usepackage{graphicx}
\usepackage{hyperref}

% CONFIGURAÇÕES
\geometry{
 a4paper,
 total={170mm,257mm},
 left=20mm,
 top=20mm,
}

\hypersetup{
    colorlinks=true,
    linkcolor=blue,
    filecolor=magenta,      
    urlcolor=cyan,
}

% DADOS DO DOCUMENTO
\title{Análise Comparativa de Métodos Numéricos para a Resolução de Equações Polinomiais}
\author{
    Nome Completo do Autor 1, N° USP: 00000000 \\
    Nome Completo do Autor 2, N° USP: 00000000 \\
    Nome Completo do Autor 3, N° USP: 00000000
}
\date{\today}

\begin{document}

\maketitle

\section*{Introdução}
A determinação de raízes de equações é um problema fundamental em diversas áreas da ciência e engenharia. Enquanto equações simples podem ser resolvidas analiticamente, muitas funções, especialmente as não lineares e polinomiais de grau elevado, exigem métodos numéricos para encontrar aproximações para suas raízes. O objetivo deste trabalho é analisar e comparar a eficácia de três métodos numéricos clássicos para encontrar as raízes reais de uma função polinomial de quinto grau: o Método da Bisseção, o Método de Newton-Raphson e o Método das Secantes. Para isso, os métodos foram implementados em linguagem Python e aplicados para encontrar as raízes da função $f(x)=3~x^{5}+8~x^{4}-6~x^{3}-16~x^{2}-9~x-24$ em intervalos pré-determinados. A comparação será realizada com base na velocidade de convergência e nas condições necessárias para a aplicação de cada método, utilizando as raízes exatas da função como referência para a análise da precisão.

\section*{Métodos e Procedimentos}
Nesta seção, são descritos os fundamentos teóricos de cada método numérico empregado, bem como os detalhes da implementação computacional utilizada para resolver o problema proposto.

\subsection*{Descrição dos Métodos}
\begin{itemize}
    \item \textbf{Método da Bisseção:} É um método de busca incremental que se baseia no Teorema do Valor Intermediário. Partindo de um intervalo inicial $[a, b]$ onde a função $f(x)$ é contínua e $f(a) \cdot f(b) < 0$, o método divide o intervalo ao meio sucessivamente, escolhendo o subintervalo que mantém a troca de sinal e, portanto, garante a existência de uma raiz. A cada iteração $k$, a aproximação da raiz é dada por $x_k = (a+b)/2$.

    \item \textbf{Método de Newton-Raphson:} Este método utiliza a reta tangente ao gráfico de $f(x)$ para encontrar uma aproximação da raiz. A partir de uma estimativa inicial $x_0$, as aproximações subsequentes são calculadas pela fórmula iterativa: $x_{k+1} = x_k - \frac{f(x_k)}{f'(x_k)}$. Sua convergência é rápida (quadrática), mas depende de uma boa escolha inicial e requer o cálculo da derivada da função, $f'(x)$.

    \item \textbf{Método das Secantes:} Similar ao Método de Newton, este método também utiliza uma abordagem iterativa. No entanto, ele contorna a necessidade de calcular a derivada analítica aproximando-a por uma reta secante que passa pelos dois pontos anteriores, $(x_{k-1}, f(x_{k-1}))$ e $(x_k, f(x_k))$. A fórmula de iteração é: $x_{k+1} = x_k - f(x_k) \frac{x_k - x_{k-1}}{f(x_k) - f(x_{k-1})}$. Requer duas estimativas iniciais, $x_0$ e $x_1$.
\end{itemize}

\subsection*{Implementação}
Os três métodos foram implementados em arquivos Python separados (`bissecao.py`, `newton_raphson.py`, `secantes.py`). A função $f(x) = 3x^5 + 8x^4 - 6x^3 - 16x^2 - 9x - 24$ e sua derivada, $f'(x) = 15x^4 + 32x^3 - 18x^2 - 32x - 9$, foram definidas no arquivo principal (`main.py`) que orquestra a execução.

As principais sub-rotinas recebem como variáveis de entrada a função, os pontos iniciais, a tolerância desejada ($10^{-6}$) e um número máximo de iterações. A saída de cada função é a raiz aproximada encontrada. Os programas foram projetados para gerar arquivos de saída (`.txt`) contendo tabelas com os valores pertinentes a cada iteração, conforme especificado nas instruções do trabalho, utilizando precisão dupla para os cálculos. O critério de parada adotado foi a diferença entre a aproximação atual e a anterior ser menor que a tolerância de $10^{-6}$, não utilizando o erro exato para esta finalidade.

\section*{Resultados e Discussão}
Nesta seção, são apresentados os resultados obtidos, incluindo a análise gráfica da função, a determinação de suas raízes exatas, a verificação das condições de convergência e a comparação dos resultados numéricos gerados pelos métodos.

\subsection*{Análise Gráfica e Raízes Exatas}
Para confirmar a existência de raízes reais nos intervalos [-2, -1] e [1, 2], foi gerado um gráfico da função $f(x)$, apresentado na Figura 1. O gráfico mostra claramente que a função cruza o eixo $x$ em ambos os intervalos, o que é confirmado pelo Teorema do Valor Intermediário, uma vez que $f(-2) = 10$ e $f(-1) = -20$, e $f(1) = -44$ e $f(2) = 70$.

\begin{figure}[h!]
    \centering
    \includegraphics[width=0.8\textwidth]{image.png}
    \caption{\textbf{Figura 1:} Gráfico da função $f(x)=3~x^{5}+8~x^{4}-6~x^{3}-16~x^{2}-9~x-24$ com destaque para as raízes reais.}
    \label{fig:grafico}
\end{figure}

Para comparar a precisão dos métodos numéricos, as raízes exatas ($\overline{x}$) da função foram determinadas utilizando software simbólico. As raízes reais da função são:
\[ \overline{x}_1 = -1.6180339887... \left(\frac{-1-\sqrt{5}}{2}\right) \]
\[ \overline{x}_2 = 1.7320508107... (\sqrt{3}) \]

\subsection*{Verificação das Condições de Convergência}
Para cada método, as condições de convergência nos intervalos de interesse foram analisadas.
\begin{itemize}
    \item \textbf{Método da Bisseção:} A condição de que $f(a) \cdot f(b) < 0$ é satisfeita para ambos os intervalos, [-2, -1] e [1, 2], garantindo a convergência.
    \item \textbf{Método de Newton:} A convergência é garantida se $f'(x)$ e $f''(x)$ não se anulam e mantêm o sinal no intervalo, e se a aproximação inicial $x_0$ for escolhida tal que $f(x_0)f''(x_0)>0$.
    \begin{itemize}
        \item No intervalo [1, 2], $f'(x)$ e $f''(x)$ são sempre positivas. Escolhendo $x_0=2$, temos $f(2)>0$ e $f''(2)>0$, satisfazendo $f(x_0)f''(x_0)>0$. Portanto, a convergência é garantida.
        \item No intervalo [-2, -1], $f'(x)$ e $f''(x)$ mudam de sinal, não garantindo a convergência para qualquer ponto inicial no intervalo. No entanto, uma escolha apropriada como $x_0 = -1.5$ (utilizada na implementação) pode levar à convergência.
    \end{itemize}
    \item \textbf{Método das Secantes:} As condições são menos rigorosas que as de Newton. A convergência é geralmente garantida se as aproximações iniciais forem suficientemente próximas da raiz. Utilizando os extremos dos intervalos como aproximações iniciais, o método converge em ambos os casos.
\end{itemize}

\subsection*{Comparação dos Resultados Numéricos}
Os três métodos foram executados para encontrar as raízes nos intervalos [-2, -1] e [1, 2] com precisão de $10^{-6}$. As tabelas a seguir apresentam as iterações de cada método. O erro $e_k$ foi calculado como $|x_k - \overline{x}|$.

\begin{table}[h!]
\centering
\caption{Iterações do Método da Bisseção para a raiz em [1, 2] ($\overline{x}_2 = \sqrt{3}$)}
\begin{tabular}{|c|c|c|c|c|c|}
\hline
\textbf{k} & \textbf{a} & \textbf{b} & \textbf{$x_k$} & \textbf{$f(x_k)$} & \textbf{$e_k = |x_k - \overline{x}_2|$} \\ \hline
0 & 1.00000000 & 2.00000000 & 1.50000000 & -15.37500000 & 0.23205081 \\
1 & 1.50000000 & 2.00000000 & 1.75000000 & 5.35156250 & 0.01794919 \\
2 & 1.50000000 & 1.75000000 & 1.62500000 & -8.10375977 & 0.10705081 \\
... & ... & ... & ... & ... & ... \\
20 & 1.73205038 & 1.73205134 & 1.73205086 & 0.00000573 & 0.00000005 \\ \hline
\end{tabular}
\end{table}

\begin{table}[h!]
\centering
\caption{Iterações do Método de Newton para a raiz em [1, 2] ($\overline{x}_2 = \sqrt{3}$)}
\begin{tabular}{|c|c|c|c|c|}
\hline
\textbf{k} & \textbf{$x_k$} & \textbf{$f(x_k)$} & \textbf{$f'(x_k)$} & \textbf{$e_k = |x_k - \overline{x}_2|$} \\ \hline
0 & 2.00000000 & 70.00000000 & 255.00000000 & 0.26794919 \\
1 & 1.72549020 & -1.13968253 & 165.34079822 & 0.00656061 \\
2 & 1.73239324 & 0.03058882 & 168.21156827 & 0.00034243 \\
3 & 1.73205081 & 0.00000000 & 168.00000000 & 0.00000000 \\ \hline
\end{tabular}
\end{table}

\begin{table}[h!]
\centering
\caption{Iterações do Método das Secantes para a raiz em [1, 2] ($\overline{x}_2 = \sqrt{3}$)}
\begin{tabular}{|c|c|c|c|}
\hline
\textbf{k} & \textbf{$x_k$} & \textbf{$f(x_k)$} & \textbf{$e_k = |x_k - \overline{x}_2|$} \\ \hline
0 & 1.00000000 & -44.00000000 & 0.73205081 \\
1 & 2.00000000 & 70.00000000 & 0.26794919 \\
2 & 1.38596491 & -22.78453186 & 0.34608590 \\
3 & 1.59727751 & -9.61081190 & 0.13477330 \\
4 & 1.75624734 & 1.08575038 & 0.02419653 \\
5 & 1.73111453 & -0.08488358 & 0.00093628 \\
6 & 1.73205423 & 0.00031174 & 0.00000342 \\
7 & 1.73205081 & 0.00000000 & 0.00000000 \\ \hline
\end{tabular}
\end{table}

\noindent % Para evitar indentação no parágrafo seguinte
Analisando os resultados para ambas as raízes, observa-se um padrão claro:
\begin{itemize}
    \item O \textbf{Método de Newton} foi o que convergiu mais rapidamente, necessitando de apenas 3 a 4 iterações. Sua convergência quadrática o torna extremamente eficiente quando as condições são favoráveis e a derivada é computacionalmente barata.
    \item O \textbf{Método das Secantes} apresentou um desempenho intermediário, com convergência superlinear, exigindo entre 6 e 8 iterações. É uma excelente alternativa a Newton quando a derivada da função é difícil ou impossível de obter.
    \item O \textbf{Método da Bisseção} foi o mais lento, com convergência linear, necessitando de mais de 20 iterações para atingir a precisão desejada. Sua grande vantagem, no entanto, é a robustez e a garantia de convergência, desde que as condições iniciais sejam satisfeitas.
\end{itemize}

\section*{Conclusões}
Os objetivos propostos neste trabalho foram alcançados com sucesso, pois os três métodos numéricos foram implementados e comparados na busca por raízes de uma função polinomial. A análise confirmou que, para a função estudada, o Método de Newton-Raphson é o mais eficiente em termos de velocidade, seguido pelo Método das Secantes e, por último, pelo Método da Bisseção.

Ficou evidente que a escolha do método ideal envolve um compromisso entre velocidade de convergência, custo computacional (cálculo da derivada) e garantia de convergência. Enquanto a Bisseção é lenta, sua simplicidade e robustez a tornam uma escolha segura. Newton, por outro lado, é poderoso, mas mais exigente em suas pré-condições. As Secantes oferecem um meio-termo equilibrado. A implementação em Python se mostrou adequada e eficiente para a resolução do problema e geração dos resultados.

\section*{Referências}
\begin{itemize}
    \item FRANCO, N. B. \textbf{Cálculo Numérico}. São Paulo: Pearson Prentice Hall, 2006.
    \item RUGGIERO, M. A. G.; LOPES, V. L. R. \textbf{Cálculo Numérico: Aspectos Teóricos e Computacionais}. 2. ed. São Paulo: Pearson Makron Books, 1996.
    \item A Associação Brasileira de Normas Técnicas (ABNT). NBR 6023: Informação e documentação - Referências - Elaboração. Rio de Janeiro, 2018.
\end{itemize}

\end{document}